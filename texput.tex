\title{\href{http://www.w3.org/html/}{HTML}}

\maketitle
\tableofcontents

\lstset{inputencoding=utf8/latin1,language=HTML,basicstyle=\footnotesize\ttfamily}

\newcommand{\htmloutput}[1]{%
\lstinputlisting{#1.html}%
\ifx \HCode\Undfef%
\else%
\HCode{%
  <iframe src="#1.html">%
  </iframe>%
}%
\fi%
}

% http://beej.us/blog/data/html5-canvas/
% http://beej.us/blog/data/html5s-canvas-2-pixel/

\chapter{Introduction}
%{{{

\section{\href{http://en.wikipedia.org/wiki/HTML}{HTML (HyperText Markup Language)}}
%{{{  

\begin{itemize}
\item HyperText Markup Language is the standard
  \href{http://en.wikipedia.org/wiki/Markup_language}{markup language}
  used to create Web pages.
\item Web browsers read HTML files and compose them into visible
  or audible Web pages.
\item It was created mainly by the physicist
  \href{http://en.wikipedia.org/wiki/Tim_Berners-Lee}{Tim Berners-Lee}
  at \href{http://en.wikipedia.org/wiki/CERN}{CERN} in the 80, which
  also wrote the first HTML interpreter (Web browser) and the first
  HTTP server (\href{http://en.wikipedia.org/wiki/CERN_httpd}{\tt CERN
    httpd}) for the
  \href{http://en.wikipedia.org/wiki/NeXT_Computer}{NeXT Computer}.
\item The first Web page was hosted by \url{http://info.cern.ch} in
  1992.
\end{itemize}

%}}}

\section{\href{http://en.wikipedia.org/wiki/XHTML}{XHTML (Extensible HTML)}}
%{{{

\begin{itemize}

\item XHTML is a reformulation of HTML in XML syntax. Example:
\begin{lstlisting}
<img src="funfun.jpg"/> <!-- This is XHTML -->
<img src="funfun.jpg">  <!-- This is HTML  -->
\end{lstlisting}

\item XHTML pages can be parsed using standard XML parsers.

\item HTML documents should be server with a
\begin{lstlisting}
Content-Type: text/html
\end{lstlisting}
MIME type HTTP line header.

\item XHTML documents with a
\begin{lstlisting}
Content-Type: text/xhtml+xml
\end{lstlisting}
or
\begin{lstlisting}
Content-Type: text/xml
\end{lstlisting}

\end{itemize}

%}}}

\section{\href{http://en.wikipedia.org/wiki/XML}{XML (Extensible Markup Language)}}
%{{{

\begin{itemize}
\item Defines a set of rules for encoding documents in a format which
  is both human-readable and machine-readable.
\end{itemize}

%}}}

\section{\href{http://en.wikipedia.org/wiki/Cascading_Style_Sheets}{CSS (Cascading Style Sheets)}}
%{{{
\begin{itemize}
\item It is designed primary to separate the document content from its
  presentation (layout, colors, fonts, etc.).
\item Define the look and layout of text and other material of the Web
  pages (in general of any markup language).
\end{itemize}
%}}}

%}}}

\chapter{\href{http://www.w3schools.com/html/html_head.asp}{Structure of a Web page}}
%{{{

\htmloutput{structure}

%}}}

\chapter{Using CSS}
%{{{ 
\section{Inline styling}
\htmloutput{inline-css}

\section{\href{internal-css.html}{Internal styling}}
\htmloutput{internal-css}

\section{\href{externla-css}{External styling}}
\lstinputlisting{external-css.css}
\htmloutput{external-css}

%}}}

\chapter{\href{http://www.w3schools.com/html/html_headings.asp}{Headings}}
%{{{

\htmloutput{headings}

%}}}

\chapter{\href{http://www.w3schools.com/html/html_links.asp}{URL links}}
%{{{

\htmloutput{URL-links}

%}}}

\chapter{\href{http://www.w3schools.com/tags/tag_hr.asp}{Horizontal rules}}
%{{{

\htmloutput{horizontal-rule}

%}}}

\chapter{\href{http://www.w3schools.com/tags/tag_pre.asp}{Indented text}}
%{{{

\htmloutput{indented-text}

%}}}

\chapter{Colorizing}
%{{{

\begin{itemize}
\item \href{http://www.w3schools.com/html/html_colornames.asp}{HTML
    color names}.
\item \href{http://www.w3schools.com/html/html_colorvalues.asp}{HTML
    color values}.
\item \href{http://www.w3schools.com/html/html_colors.asp}{HTML
    color shades}.
\end{itemize}

\htmloutput{using-color}

%}}}

\chapter{\href{http://www.w3schools.com/cssref/css_websafe_fonts.asp}{Font selection}}
%{{{

\htmloutput{font-selection}

%}}}

\chapter{\href{http://www.w3schools.com/html/html5_browsers.asp}{Creating new elements}}
%{{{

\htmloutput{creating-elements}

%}}}

\chapter{\href{http://www.w3schools.com/charsets/ref_html_8859.aspl}{Writting in Spanish}}
%{{{

\htmloutput{spanish}

%}}}

\chapter{\href{http://www.w3schools.com/tags/att_font_size.asp}{Font size}}
%{{{

\htmloutput{font-size}

%}}}

\chapter{\href{http://www.w3schools.com/cssref/pr_text_text-align.asp}{Text alignment}}
%{{{

\htmloutput{text-alignment}

%}}}

\chapter{\href{http://www.w3schools.com/html/html_formatting.asp}{Text style}}
%{{{

\htmloutput{text-style}

%}}}

\chapter{\href{http://www.w3schools.com/tags/tag_br.asp}{Line breaks}}
%{{{

\htmloutput{line-break}

%}}}

\chapter{\href{http://www.w3schools.com/html/html_tables.asp}{Creating tables}}
%{{{

\htmloutput{table}

%}}}

\chapter{\href{http://www.w3schools.com/html/html_iframe.asp}{Inserting Web pages}}
%{{{

\htmloutput{iframe}

%}}}

\chapter{\href{http://www.w3schools.com/tags/tag_audio.asp}{Playing audio}}
%{{{

\htmloutput{audio}

%}}}

\chapter{\href{http://www.w3schools.com/tags/tag_video.asp}{Playing video}}
%{{{

\htmloutput{video}

%}}}

\chapter{\href{http://www.w3schools.com/tags/tag_img.asp}{Displaying images}}
%{{{

\htmloutput{images}

%}}}

\chapter{\href{http://www.w3schools.com/tags/tag_figure.asp}{Inserting figures}}
%{{{

\htmloutput{figures}

%}}}

\chapter{\href{http://www.w3schools.com/html/html_scripts.asp}{Using JavaScript}}
%{{{

\begin{itemize}
\item Useful to create ``live'' Web pages.
\item \href{http://www.w3schools.com/js}{A tutorial}.
\end{itemize}

\section{\href{http://www.w3schools.com/tags/tag_button.asp}{Creating Buttons}}
%{{{

\htmloutput{button}

%}}}

\section{\href{http://www.w3schools.com/tags/tag_canvas.asp}{Drawing figures}}
%{{{

\htmloutput{filled-rectangle}

%}}}

\section{\href{A more complex JavaScript demo}{contrast-sensitivity-image.html}: \href{http://en.wikipedia.org/wiki/Contrast_(vision)}{The Contrast Sensitivity Function (Campbell and Robson CSF test chart)}}
%{{{

\lstinputlisting{contrast-sensitivity-image.html}%
\ifx \HCode\Undfef%
\else%
\HCode{
  <iframe width="800" height="600" src="contrast-sensitivity-image.html">%
  </iframe>
}
\fi

%}}}

%}}}

